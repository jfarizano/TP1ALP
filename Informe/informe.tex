\documentclass[11pt, fleqn]{article}
\usepackage[a4paper, margin=2.54cm]{geometry}
\usepackage[utf8]{inputenc}
\usepackage[spanish, mexico]{babel}
\usepackage[spanish]{layout}
\usepackage[article]{ragged2e}
\usepackage{textcomp}
\usepackage{amsmath}
\usepackage{amssymb}
\usepackage{amsfonts}
\usepackage{proof}

% Macros para type-setear el Lenguaje Imperativo Simple del TP1 de ALP.

\usepackage{amsmath, amssymb, proof}

%%%%%%%%%%%%%%%%%%%%%%%%%%%%%%%%%%%%%%%%%%%%%%%%%%%%%%%%%%%%%%%%%%%%%%%%%%
% Sintaxis Abstracta LIS
%%%%%%%%%%%%%%%%%%%%%%%%%%%%%%%%%%%%%%%%%%%%%%%%%%%%%%%%%%%%%%%%%%%%%%%%%%

%% Expresiones enteras
\newcommand{\euminus}   [1]  {-_{u} #1}
\newcommand{\eplus}     [2]  {#1 + #2}
\newcommand{\eminus}    [2]  {#1 -_{b} #2}
\newcommand{\etimes}    [2]  {#1 \times #2}
\newcommand{\ediv}      [2]  {#1 \div #2}
\newcommand{\eseq}      [2]  {#1 ,~ #2}
\newcommand{\eassgn}    [2]  {#1 = #2}

%% Expresiones booleanas
\newcommand{\etrue}          {\textbf{true}}
\newcommand{\efalse}         {\textbf{false}}
\newcommand{\eeq}       [2]  {#1 == #2}
\newcommand{\eneq}      [2]  {#1 \neq #2}
\newcommand{\elt}       [2]  {#1 < #2}
\newcommand{\egt}       [2]  {#1 > #2}
\newcommand{\eand}      [2]  {#1 \wedge #2}
\newcommand{\eor}       [2]  {#1 \vee #2}
\newcommand{\enot}      [1]  {\neg #1}

%% Comandos
\newcommand{\cskip}          {\textbf{skip}}
\newcommand{\clet}      [2]  {#1 = #2}
\newcommand{\cseq}      [2]  {#1;~#2}
\newcommand{\cif}       [3]  {\textbf{if}~#1~\textbf{then}~#2~\textbf{else}~#3}
\newcommand{\cwhile}    [2]  {\textbf{while}~#1~\textbf{do}~#2}
\newcommand{\cfor}      [4]  {\textbf{for}(#1;~#2~;#3)~#4}

%%%%%%%%%%%%%%%%%%%%%%%%%%%%%%%%%%%%%%%%%%%%%%%%%%%%%%%%%%%%%%%%%%%%%%%%%%
% Operadores del metalenguaje (de enteros y booleanos)
%%%%%%%%%%%%%%%%%%%%%%%%%%%%%%%%%%%%%%%%%%%%%%%%%%%%%%%%%%%%%%%%%%%%%%%%%%

%% Símbolos en negrita
\newcommand{\bs}        [1]  {\boldsymbol{#1}}

%% Números en negrita
\newcommand{\bm}        [1]  {\mathbf{#1}}

%% Operaciones sobre enteros
\newcommand{\muminus}   [1]  {\bs{-} #1}
\newcommand{\mplus}     [2]  {#1 \bs{+} #2}
\newcommand{\mminus}    [2]  {#1 \bs{-} #2}
\newcommand{\mtimes}    [2]  {#1 \bs{\times} #2}
\newcommand{\mdiv}      [2]  {#1 \bs{\div} #2}

%% Operaciones sobre booleanos
\newcommand{\mtrue}          {\textbf{true}}
\newcommand{\mfalse}         {\textbf{false}}
\newcommand{\meq}       [2]  {#1 \bs{=} #2}
\newcommand{\mneq}      [2]  {#1 \bs{\neq} #2}
\newcommand{\mlt}       [2]  {#1 \bs{<} #2}
\newcommand{\mgt}       [2]  {#1 \bs{>} #2}
\newcommand{\mand}      [2]  {#1 \bs{\wedge} #2}
\newcommand{\mor}       [2]  {#1 \bs{\vee} #2}
\newcommand{\mnot}      [1]  {\bs{\neg} #1}

%%%%%%%%%%%%%%%%%%%%%%%%%%%%%%%%%%%%%%%%%%%%%%%%%%%%%%%%%%%%%%%%%%%%%%%%%%
% Resto del metalenguaje
%%%%%%%%%%%%%%%%%%%%%%%%%%%%%%%%%%%%%%%%%%%%%%%%%%%%%%%%%%%%%%%%%%%%%%%%%%

%% Pares de la forma <a, b>
\newcommand{\pair}      [2]  {\left\langle #1,~#2\right\rangle}

%% Relación semántica big-step de expresiones
\newcommand{\sexp}      [2]  {#1 \Downarrow_{\mathrm{exp}} #2}

%% Relación semántica small-step de expresiones
\newcommand{\step}      [2]  {#1 \rightsquigarrow #2}

%% Clausura reflexo-transitiva de \step
\newcommand{\steps}     [2]  {#1 \rightsquigarrow^{*} #2}

%% Extensión funcional para estado - [f| x: e]
\newcommand{\extf}      [3]  {\left[ #1|~#2:~#3 \right]}

%% Regla de inferencia, con argumentos invertidos
%% \infr{L}{A}{B} =     A
%%                  --------- L
%%                      B
\newcommand{\infr}    [3] [] {\infer[\textsc{#1}]{#3}{#2}}

%% Estilo para nombre de raglas
\newcommand{\rn}        [1]  {\textsc{#1}}

%%%%%%%%%%%%%%%%%%%%%%%%%%%%%%%%%%%%%%%%%%%%%%%%%%%%%%%%%%%%%%%%%%%%%%%%%%
% Nombre de reglas
%%%%%%%%%%%%%%%%%%%%%%%%%%%%%%%%%%%%%%%%%%%%%%%%%%%%%%%%%%%%%%%%%%%%%%%%%%
\newcommand{\rNVal}          {\rn{NVal}}
\newcommand{\rVar}           {\rn{Var}}
\newcommand{\rUMinus}        {\rn{UMinus}}
\newcommand{\rPlus}          {\rn{Plus}}
\newcommand{\rMinus}         {\rn{Minus}}
\newcommand{\rTimes}         {\rn{Times}}
\newcommand{\rDiv}           {\rn{Div}}
\newcommand{\rEq}            {\rn{Eq}}
\newcommand{\rNEq}           {\rn{NEq}}
\newcommand{\rGt}            {\rn{Gt}}
\newcommand{\rLt}            {\rn{Lt}}
\newcommand{\rBVal}          {\rn{BVal}}
\newcommand{\rNot}           {\rn{Not}}
\newcommand{\rOr}            {\rn{Or}}
\newcommand{\rAnd}           {\rn{And}}

\newcommand{\rAss}           {\rn{Ass}}
\newcommand{\rSeqA}          {\rn{Seq$_1$}}
\newcommand{\rSeqB}          {\rn{Seq$_2$}}
\newcommand{\rIfA}           {\rn{If$_1$}}
\newcommand{\rIfB}           {\rn{If$_2$}}
\newcommand{\rWhileA}        {\rn{While$_1$}}
\newcommand{\rWhileB}        {\rn{While$_2$}}
\usepackage{bussproofs}
\usepackage{graphicx}

\setlength{\parindent}{5pt}

\title{Trabajo Práctico 1}
\author{Mellino, Natalia \and Farizano, Juan Ignacio}
\date{}

\begin{document}
\maketitle

%==============================================================================
%==============================================================================
%==============================================================================

\section*{Ejercicio 1}
A continuación extendemos las sintaxis abstracta y concreta para incluir asignaciones
de variables como expresiones enteras y para escribir una secuencia de expresiones enteras.

\subsection*{Sintaxis Abstracta}
\begin{align*}
intexp ::=& \; nat \; | \; var \; | \; -_u \; intexp \\
          &| \;\; intexp \; +  \; intexp \\
          &| \;\; intexp \; -_b  \; intexp \\
          &| \;\; intexp \; \times  \; intexp \\
          &| \;\; intexp \; \div  \; intexp \\
          &| \;\; var = intexp \\
          &| \;\; intexp, \; intexp \\
boolesxp ::=& \; \textbf{true} \; | \; \textbf{false} \\
            &| \;\; intexp \; ==  \; intexp \\
            &| \;\; intexp \; \neq  \; intexp \\
            &| \;\; intexp \; <  \; intexp \\
            &| \;\; intexp \; >  \; intexp \\
            &| \;\; boolexp \; \lor  \; boolexp \\
            &| \;\; boolexp \; \land  \; boolexp \\
            &| \;\; \neg \; boolexp \\
comm ::=& \; \textbf{skip} \\
        &| \;\; var = intexp \\
        &| \;\; comm; \; comm \\
        &| \;\; \textbf{if} \; boolexp \; \textbf{then} \; comm \; \textbf{else} \; comm \\
        &| \; \; \textbf{while} \; boolexp \; \textbf{do} \; comm
\end{align*}

%==============================================================================

\subsection*{Sintaxis Concreta}

\begin{align*}
digit ::=& \; \text{'0'} \; | \; \text{'1'} | \; \dotsc \; | \; \text{'9'} \\
letter ::=& \; \text{'a'} \; | \; \dotsc \; | \; \text{'Z'} \\
nat ::=& \; digit \; | \; digit \; nat \\
var ::=& \; letter \; | \; letter \; var \\
intexp ::=& \; nat \\
          &| \; var \\
          &| \; \text{'-'} \; intexp \\
          &| \;\; intexp \; \text{'+'}  \; intexp \\
          &| \;\; intexp \; \text{'-'}  \; intexp \\
          &| \;\; intexp \; \text{'*'}  \; intexp \\
          &| \;\; intexp \; \text{'/'}  \; intexp \\
          &| \;\; \text{'('} \; intexp \; \text{')'} \\
          &| \;\; var \; \text{'='} \; intexp \\
          &| \;\; intexp \; \text{','} \; intexp \\
boolesxp ::=& \; \text{'true'} \; | \; \text{'false'} \\
            &| \;\; intexp \; \text{'=='} \; intexp \\
            &| \;\; intexp \; \text{'!='}  \; intexp \\
            &| \;\; intexp \; \text{'$<$'}  \; intexp \\
            &| \;\; intexp \; \text{'$>$'}  \; intexp \\
            &| \;\; boolexp \; \text{'\&\&'}  \; boolexp \\
            &| \;\; boolexp \; \text{'$\vert\vert$'}  \; boolexp \\
            &| \;\; \text{'!'} \; boolexp \\
            &| \;\; \text{'('} \; boolexp \; \text{')'} \\
com ::=& \; \textbf{skip} \\
       &| \;\; var \; \text{'='} \; intexp \\
       &| \;\; comm \; \text{';'} \; comm \\
       &| \;\; \text{'if'} \;\;boolexp \;\; \text{'\{'} \;\; comm \;\; \text{'\}'} \\
       &| \;\; \text{'if'} \;\;boolexp \;\; \text{'\{'} \;\; comm \;\; \text{'\}'}  \;\; \text{'else'} \;\; \text{'\{'} \;\; comm \;\; \text{'\}'} \\
       &| \;\; \text{'while'} \;\;boolexp \;\; \text{'\{'} \;\; comm \;\; \text{'\}'} \\
\end{align*}

%==============================================================================
%==============================================================================
%==============================================================================

\section*{Ejercicio 2:}
Para extender la realización de la sistaxis abstracta en Haskell, incluimos
estos contructores en el tipo de datos parametrizado Exp $a$. 
\begin{align*}
      \text{EAssgn} &:: \text{Variable} \rightarrow \text{Exp Int} \rightarrow \text{Exp Int} \\
      \text{ESeq} &:: \text{Exp Int} \rightarrow \text{Exp Int} \rightarrow \text{Exp Int} \\
\end{align*}
En el archivo \emph{src/AST.hs} se encuentra reflejado este cambio.

%==============================================================================
%==============================================================================
%==============================================================================

\section*{Ejercicio 3:}
Para implementar el parser, modificamos la gramática extendida en el ejercicio 1
en una que no presente ambigüedad.

\subsection*{Sintaxis Abstracta}
\begin{align*}
intexp ::=& \; intexp, \; intexp1 \; \vert \; intexp1 \\
intexp1 ::=& \; var = intexp1 \; \vert \; intexp2 \\
intexp2 ::=& \; intexp2 \; +  \; intexp3 \; \vert \; intexp2 \; -_b  \; intexp3 \; \vert \; intexp2\\
intexp3 ::=& \; intexp3 \; \times  \; intexp4 \; \vert \; intexp3 \; \div  \; intexp4 \; \vert \; intexp4\\
intexp4 ::=& \; -_u \; intexp4 \; \vert \; nat \; \vert \; var \; \vert \; (intexp) \\
& \\
boolexp ::=& \; boolexp \; \lor \; boolexp1 \; \vert \; boolexp1 \\
boolexp1 ::=& \; boolexp1 \; \land \; boolexp2 \; \vert \; boolexp2 \\
boolexp2 ::=& \; \neg \; boolexp2 \; \vert \; boolexp3 \\
boolexp3 ::=& \; \textbf{true} \; \vert \; \textbf{false} \\
            &| \;\; intexp == intexp \\
            &| \;\; intexp \neq intexp \\
            &| \;\; intexp > intexp \\
            &| \;\; intexp < intexp \\
            &| \;\; (boolexp) \\
&\\
comm ::=& \; comm; \; comm1 \; \vert \; comm1 \\
comm1 ::=& \; \textbf{skip} \\
         &| \;\; var = intexp \\
         &| \;\; \textbf{if} \; boolexp \; \textbf{then} \; comm \; \textbf{else} \; comm \\
         &| \; \; \textbf{while} \; boolexp \; \textbf{do} \; comm
\end{align*}

%==============================================================================

\subsection*{Sintaxis Concreta}

\begin{align*}
digit ::=& \; \text{'0'} \; | \; \text{'1'} | \; \dotsc \; | \; \text{'9'} \\
letter ::=& \; \text{'a'} \; | \; \dotsc \; | \; \text{'Z'} \\
nat ::=& \; digit \; | \; digit \; nat \\
var ::=& \; letter \; | \; letter \; var \\
&\\
intexp ::=& \; intexp \; \text{','} \; intexp1 \; \vert \; intexp1 \\
intexp1 ::=& \; var \; \text{'='} \; intexp1 \; \vert \; intexp2 \\
intexp2 ::=& \; intexp2 \; \text{'+'}  \; intexp3 \; \vert \; intexp2 \; \text{'-'} \; intexp3 \; \vert \; intexp2\\
intexp3 ::=& \; intexp3 \; \text{'*'}  \; intexp4 \; \vert \; intexp3 \; \text{'/'} \; intexp4 \; \vert \; intexp4\\
intexp4 ::=& \; \text{'-'} \; intexp4 \; \vert \; nat \; \vert \; var \; \vert \; \text{'('} intexp \text{')'} \\
& \\
boolexp ::=& \; boolexp \; \text{'$\vert\vert$'} \; boolexp1 \; \vert \; boolexp1 \\
boolexp1 ::=& \; boolexp1 \; \text{'\&\&'} \; boolexp2 \; \vert \; boolexp2 \\
boolexp2 ::=& \; \text{'!='} \; boolexp2 \; \vert \; boolexp3 \\
boolexp3 ::=& \; \textbf{true} \; \vert \; \textbf{false} \\
            &| \;\; intexp \; \text{'=='} \; intexp \\
            &| \;\; intexp \; \text{'!='} \; intexp \\
            &| \;\; intexp \; \text{'$<$'} \; intexp \\
            &| \;\; intexp \; \text{'$>$'} \; intexp \\
            &| \;\; (boolexp) \\
&\\
comm ::=& \; comm \; \text{';'} \; comm1 \; \vert \; comm1 \\
comm1 ::=& \; \textbf{skip} \\
         &| \;\; var \; \text{'='} \; intexp \\
         &| \;\; \text{'if'} \;\;boolexp \;\; \text{'\{'} \;\; comm \;\; \text{'\}'} \\
         &| \;\; \text{'if'} \;\;boolexp \;\; \text{'\{'} \;\; comm \;\; \text{'\}'}  \;\; \text{'else'} \;\; \text{'\{'} \;\; comm \;\; \text{'\}'} \\
         &| \;\; \text{'while'} \;\;boolexp \;\; \text{'\{'} \;\; comm \;\; \text{'\}'} \\
\end{align*}

%==============================================================================
%==============================================================================
%==============================================================================

\section*{Ejercicio 4:}

A continuación modificamos la semántica big-step para incluir la asignación de variables como
expresiones y para las secuencias de expresiones.

\[
\infr[\rn{EAssgn}]
  {
    \sexp{\pair{e}{\sigma}}{\pair{n}{\sigma'}}
  }
  {
    \sexp{\pair{\eassgn{var}{e}}{\sigma}}{\pair{n}{\extf{\sigma'}{var}{n}}}
  }
\
\]
\smallskip
\[
\infr[\rn{ESeq}]
  {
    \sexp{\pair{e_0}{\sigma}}{\pair{n_0}{\sigma'}}
    \quad
    \sexp{\pair{e_1}{\sigma'}}{\pair{n_1}{\sigma''}}
  }
  {
    \sexp{\pair{\eseq{e_0}{e_1}}{\sigma}}{\pair{n_1}{\sigma''}}
  }
\]

%==============================================================================
%==============================================================================
%==============================================================================

\section*{Ejercicio 5:}

Asumimos que la relación $\Downarrow_{exp}$ es determinista y procedemos por inducción
en la última regla de derivación.
Queremos probar : $ c \rightsquigarrow c', \: c \rightsquigarrow c'' \Rightarrow c' = c'' $ \\

$\bullet$ Si $ c \rightsquigarrow c' $ usando como última regla \rn{ASS}: \\
$c$ tiene la forma $\langle v=e, \sigma \rangle$ y tenemos la premisa
$\langle e, \sigma \rangle \: \Downarrow_{exp} \: \langle n, \sigma' \rangle$,
inmediatamente debido a la regla \rn{Asumimos} obtenemos que
$c' = \: \langle \textbf{skip}, [\sigma' \: \vert v : n] \rangle$.

Supongamos entonces, que esta relación no es determinista, es decir que $ c' \neq c'' $.
Por la forma que tiene $c$ observemos que la única regla que podemos usar en la
derivación $ c \rightsquigarrow c'' $ es la regla \rn{Asumimos}, entonces tenemos que:
$ \langle v=e, \sigma \rangle \rightsquigarrow \langle \textbf{skip}, [\sigma'' \: \vert v : n'] \rangle$
con la premisa $\langle e, \sigma \rangle \: \Downarrow_{exp} \: \langle n', \sigma'' \rangle$
Como $ c' \neq c'' $ vemos que $ \sigma' \neq \sigma'' \lor n \neq n''$.
Esto es una contradicción ya que por determinismo de la relación $\Downarrow_{exp}$ 
necesariamente debe ocurrir que $ \sigma' = \sigma'' \land n = n''$. \\

$\therefore \; c' = c''$ \\


%==============================================================================

$\bullet$ Si $ c \rightsquigarrow c' $ usando como última regla \rn{SEQ1}: \\
$c$ tiene la forma $\langle \textbf{skip};c_1, \sigma \rangle$ y tenemos que $c' = \langle c_1, sigma \rangle$.
Por la forma que tiene $c$, observemos que no es posible aplicar ninguna otra regla
de inferencia, por lo tanto en la derivación $ c \rightsquigarrow c'' $ se tiene que
$c'' = \langle c_1, \sigma \rangle$. \\

$\therefore \; c' = c''$ \\

%==============================================================================

$\bullet$ Si $ c \rightsquigarrow c' $ usando como última regla \rn{SEQ2}: \\
Tenemos entonces que $c$ tiene la forma: $\langle c_0;c_1, \sigma \rangle$ y
además:  $ \langle c_0, \; \sigma \rangle \rightsquigarrow \langle c_0', \sigma' \rangle$.
Ahora, analicemos que pasa en $ c \rightsquigarrow c'' $: ¿qué reglas podemos aplicar?

\begin{itemize}
      \item No podemos aplicar \rn{SEQ1} ya que esto nos diría que $c_0 = \textbf{skip}$ y esto
            nos contradice $ \langle c_0, \sigma \rangle \rightsquigarrow \langle c_0', \; \sigma \rangle$ ya que
            si $c_0 = skip$ el estado $\sigma$ no debería pasar a ser $\sigma'$.
      \item En cuanto a las demás reglas, es claro que por la forma que tiene $c$ no es
            posible aplicarlas.
\end{itemize}

Por lo tanto, la única regla que podemos aplicar es \rn{SEQ2}. Si suponemos que 
$c' \neq c''$ tenemos entonces que $c$ tiene la forma: $\langle c_0;c_1,\sigma \rangle$ y
además:  $ \langle c_0, \sigma \rangle \rightsquigarrow \langle c_0'', \sigma'' \rangle$.
Luego sigue que $c''$ tiene la forma: $ \langle c_0'';c1, \sigma'' \rangle$.
Sin embargo, observemos que $ \langle c_0, \sigma \rangle \rightsquigarrow \langle c_0', \; \sigma' \rangle$
es una subderivación y por lo tanto vale nuestra Hipótesis Inductiva, es decir necesariamente:

\begin{equation*}
 \langle c_0, \; \sigma \rangle \rightsquigarrow \langle c_0', \sigma' \rangle = \langle c_0, \; \sigma \rangle \rightsquigarrow \langle c_0'', \; \sigma'' \rangle
\end{equation*}

Por lo tanto, $c_0' = c_0''$ y $\sigma' = \sigma''$. \\

$\therefore \; c' = c''$ \\

%==============================================================================

$\bullet$ Si $ c \rightsquigarrow c' $ usando como última regla \rn{IF1}: \\
$c$ tiene la forma $ \langle \textbf{if } \text{b \textbf{then} } c_0 \textbf{ else } c_1, \; \sigma \rangle$
y además $ \langle b, \sigma \rangle \Downarrow_{exp} \langle \textbf{true}, \sigma' \rangle$.
Entonces $c'$ tiene la forma $ \langle c_0, \sigma' \rangle$. ¿Qué pasa en el caso
de  $ c \rightsquigarrow c'' $. Sabemos que por la forma que tiene $c$ y por el hecho de
que $ \langle b, \sigma \rangle \Downarrow_{exp} \langle \textbf{true}, \sigma' \rangle$
sólo podemos aplicar la regla \rn{IF1}. Recordemos que la relación $\Downarrow_{exp}$ es
determinista y por eso podemos decir que sólo podamos aplicar \rn{IF1}. inmediatamente
de esta conclusión surge que $c' = c''$. \\

%==============================================================================

$\bullet$ Si $ c \rightsquigarrow c' $ usando como última regla \rn{IF2}:
\begin{center}(este caso es análogo a la regla \rn{IF1})\end{center}

%==============================================================================

$\bullet$ Si $ c \rightsquigarrow c' $ usando como última regla \rn{WHILE1}: tenemos entonces que
\begin{itemize}
      \item $c$ tiene la forma $ \langle \textbf{while } \text{b \textbf{do} } c_0, \sigma \rangle$.
      \item $ \langle b, \sigma \rangle \Downarrow_{exp} \langle \textbf{true}, \sigma' \rangle$
      \item $c' = \langle c0; \textbf{while } \text{b \textbf{do} } c_0, \sigma' \rangle$
\end{itemize}

Luego en $ c \rightsquigarrow c'' $ ocurre que por la forma de $c$ y por el determinismo
de $\Downarrow_{exp}$, la única regla que podemos usar es \rn{WHILE1}. Por estas dos cosas
surge inmediatamente que $c' = c''$. \\

%==============================================================================

$\bullet$ Si $ c \rightsquigarrow c' $ usando como última regla \rn{WHILE2}: tenemos entonces que
\begin{itemize}
      \item $c$ tiene la forma $ \langle \textbf{while } \text{b \textbf{do} } c_0, \; \sigma \rangle$.
      \item $ \langle b, \sigma \rangle \Downarrow_{exp} \langle \textbf{false}, \sigma' \rangle$
      \item $c' = \langle \textbf{skip}, \sigma' \rangle$
\end{itemize}

Luego en $ c \rightsquigarrow c'' $ ocurre que por la forma de $c$ y por el determinismo
de $\Downarrow_{exp}$, la única regla que podemos usar es \rn{WHILE1}. Por estas dos cosas
surge inmediatamente que $c' = c''$. \\

Por lo tanto, concluimos que la relación $ \rightsquigarrow $ es determinista.

%==============================================================================
%==============================================================================
%==============================================================================

\section*{Ejercicio 6:}

Utilizamos seis árboles distintos para realizar la derivación en varios pasos. Al
final de la misma uniremos todos los resultados en un único árbol utilizando las reglas
de la clausura reflexivo-transitiva. \\
En el primer árbol probamos $\step{t_1}{t_2}$
\begin{prooftree}
    \AxiomC{}
    \RightLabel{\rn{NVal}}
    \UnaryInfC{$
        \sexp
            {
             \pair{1}{\extf{\extf{\sigma}{x}{2}}{y}{2}}   
            }
            {
                \pair{1}{\extf{\extf{\sigma}{x}{2}}{y}{2}}
            }
    $}
    \RightLabel{\rn{EAssgn}}
    \UnaryInfC{$
        \sexp
            {
             \pair{\eassgn{y}{1}}{\extf{\extf{\sigma}{x}{2}}{y}{2}}   
            }
            {
                \pair{1}{\extf{\extf{\sigma}{x}{2}}{y}{1}}
            }
    $}
    \RightLabel{\rn{Ass}}
    \UnaryInfC{$
        \step
            {
             \pair{\clet{x}{\eassgn{y}{1}}}{\extf{\extf{\sigma}{x}{2}}{y}{2}}
            }
            {
             \pair{\cskip}{\extf{\extf{\sigma}{x}{1}}{y}{1}}
            }
    $}
    \RightLabel{\rn{Seq$_2$}}
    \UnaryInfC{$
        \step
            {
             \underbrace
             {
              \pair{\cseq{\clet{x}{\eassgn{y}{1}}}{\underbrace{\cwhile{\egt{x}{0}}{\clet{x}{\eminus{x}{y}}}}_{c_1}}}
               {\extf{\extf{\sigma}{x}{2}}{y}{2}}
             }_{t_1}
            }
            {
              \underbrace
              {
               \pair{\cseq{\cskip}{c_1}}{\extf{\extf{\sigma}{x}{1}}{y}{1}}
              }_{t_2}
            }
    $}
\end{prooftree}

En este segundo árbol probamos $\step{t_2}{t_3}$

\begin{prooftree}
    \AxiomC{}
    \RightLabel{\rn{Seq$_1$}}
    \UnaryInfC{$
        \step
            {
             \underbrace
             {
              \pair{\cseq{\cskip}{\underbrace{\cwhile{\egt{x}{0}}{\clet{x}{\eminus{x}{y}}}}_{c_1}}}
                  {\extf{\extf{\sigma}{x}{1}}{y}{1}}
             }_{t_2}
            }
            {
             \underbrace{\pair{c_1}{\extf{\extf{\sigma}{x}{1}}{y}{1}}}_{t_3}
            }
    $}
\end{prooftree}

En este tercer árbol probamos $\step{t_3}{t_4}$.

\begin{prooftree}
    \AxiomC{}
    \RightLabel{\rn{Var}}
    \UnaryInfC{$
        \sexp
            {
             \pair{x}{\sigma'}
            }
            {
             \pair{1}{\sigma'}
            }
    $}
    \AxiomC{}
    \RightLabel{\rn{NVal}}
    \UnaryInfC{$
        \sexp
            {
             \pair{0}{\sigma'}
            }
            {
             \pair{0}{\sigma'}
            }
    $}
    \RightLabel{\rn{Gt}}
    \BinaryInfC{$
        \sexp
            {
             \pair{\egt{x}{0}}{\sigma'}
            }
            {
             \pair{\etrue}{\sigma'}
            }
    $}
    \RightLabel{\rn{While$_1$}}
    \UnaryInfC{$
        \step
            {
             \underbrace
             {
              \pair{\underbrace{\cwhile{\egt{x}{0}}{\clet{x}{\eminus{x}{y}}}}_{c_1}}{\underbrace{\extf{\extf{\sigma}{x}{1}}{y}{1}}_{\sigma'}}
             }_{t_3}
            }
            {
             \underbrace
             {
              \pair{\cseq{\clet{x}{\eminus{x}{y}}}{c_1}}{\sigma'}
             }_{t_4}
            }
    $}
\end{prooftree}

En este cuarto árbol probamos $\step{t_4}{t_5}$.

\begin{prooftree}
    \AxiomC{}
    \RightLabel{\rn{Var}}
    \UnaryInfC{$
        \sexp
            {
             \pair{x}{\sigma'}
            }
            {
             \pair{1}{\sigma'}
            }
    $}
    \AxiomC{}
    \RightLabel{\rn{Var}}
    \UnaryInfC{$
        \sexp
            {
             \pair{y}{\sigma'}
            }
            {
             \pair{1}{\sigma'}
            }
    $}
    \RightLabel{\rn{Minus}}
    \BinaryInfC{$
        \step
            {
             \pair{\eminus{x}{y}}{\sigma'}
            }
            {
             \pair{0}{\sigma'}
            }
    $}
    \RightLabel{\rn{Ass}}
    \UnaryInfC{$
        \step
            {
             \pair{\clet{x}{\eminus{x}{y}}}{\sigma'}
            }
            {
             \pair{\cskip}{\sigma''}
            }
    $}
    \RightLabel{\rn{Seq$_2$}}
    \UnaryInfC{$
        \step
            {
             \underbrace
             {
              \pair{\cseq{\clet{x}{\eminus{x}{y}}}{\underbrace{\cwhile{\egt{x}{0}}{\clet{x}{\eminus{x}{y}}}}_{c_1}}}
                   {\underbrace{\extf{\extf{\sigma}{x}{1}}{y}{1}}_{\sigma'}}
             }_{t_4}
            }
            {
             \underbrace
             {
              \pair{\cseq{\cskip}{c_1}}
                   {\underbrace{\extf{\extf{\sigma}{x}{0}}{y}{1}}_{\sigma''}}
             }_{t_5}
            }
    $}
\end{prooftree}

En este quinto árbol probamos $\step{t_5}{t_6}$.

\begin{prooftree}
    \AxiomC{}
    \RightLabel{\rn{Seq$_1$}}
    \UnaryInfC{$
        \step
            {
             \underbrace
             {
              \pair{\cseq{\cskip}{\underbrace{\cwhile{\egt{x}{0}}{\clet{x}{\eminus{x}{y}}}}_{c_1}}}
                   {\extf{\extf{\sigma}{x}{0}}{y}{1}}
             }_{t_5}
            }
            {
             \underbrace
             {
              \pair{c_1}
                   {\extf{\extf{\sigma}{x}{0}}{y}{1}}
             }_{t_6}
            }
    $}
\end{prooftree}

Por último probamos $\step{t_6}{t_7}$.

\begin{prooftree}
    \AxiomC{}
    \RightLabel{\rn{Var}}
    \UnaryInfC{$
        \sexp
            {
             \pair{x}{\sigma'}
            }
            {
             \pair{0}{\sigma'}
            }
    $}
    \AxiomC{}
    \RightLabel{\rn{NVal}}
    \UnaryInfC{$
        \sexp
            {
             \pair{0}{\sigma'}
            }
            {
             \pair{0}{\sigma'}
            }
    $}
    \RightLabel{\rn{Gt}}
    \BinaryInfC{$
        \sexp
            {
             \pair{\egt{x}{0}}{\sigma'}
            }
            {
             \pair{\efalse}{\sigma'}
            }
    $}
    \RightLabel{\rn{While$_2$}}
    \UnaryInfC{$
        \step
            {
             \underbrace
             {
              \pair{\cwhile{\egt{x}{0}}{\clet{x}{\eminus{x}{y}}}}
                   {\underbrace{\extf{\extf{\sigma}{x}{0}}{y}{1}}_{\sigma'}}
             }_{t_6}
            }
            {
             \underbrace{\pair{\cskip}{\sigma'}}_{t_7}
            }
    $}
\end{prooftree}

Ahora, utilizando las siguientes reglas de la clausura reflexivo transitiva:

\[
\infr[\rn{E$_1$}]{\step{t}{t'}}{\steps{t}{t'}}
\
\qquad
\infr[\rn{E$_2$}]{\steps{t}{t'} \quad \steps{t'}{t''}}{\steps{t}{t''}}
\]

probamos que $\steps{t_1}{t_7}$:

\begin{prooftree}
    \AxiomC{$\step{t_1}{t_2}$}
    \RightLabel{\rn{E$_1$}}
    \UnaryInfC{$\steps{t_1}{t_2}$}
    \AxiomC{$\step{t_2}{t_3}$}
    \RightLabel{\rn{E$_1$}}
    \UnaryInfC{$\steps{t_2}{t_3}$}
    \RightLabel{\rn{E$_2$}}
    \BinaryInfC{$\steps{t_1}{t_3}$}
    \AxiomC{$\step{t_3}{t_4}$}
    \RightLabel{\rn{E$_1$}}
    \UnaryInfC{$\steps{t_3}{t_4}$}
    \AxiomC{$\step{t_4}{t_5}$}
    \RightLabel{\rn{E$_1$}}
    \UnaryInfC{$\steps{t_4}{t_5}$}
    \RightLabel{\rn{E$_2$}}
    \BinaryInfC{$\steps{t_3}{t_5}$}
    \RightLabel{\rn{E$_2$}}
    \BinaryInfC{$\steps{t_1}{t_5}$}
    \AxiomC{$\step{t_5}{t_6}$}
    \RightLabel{\rn{E$_1$}}
    \UnaryInfC{$\steps{t_5}{t_6}$}
    \AxiomC{$\step{t_6}{t_7}$}
    \RightLabel{\rn{E$_1$}}
    \UnaryInfC{$\steps{t_6}{t_7}$}
    \RightLabel{\rn{E$_2$}}
    \BinaryInfC{$\steps{t_5}{t_7}$}
    \RightLabel{\rn{E$_1$}}
    \BinaryInfC{$\steps{t_1}{t_7}$}
\end{prooftree}

%==============================================================================
%==============================================================================
%==============================================================================

\section*{Ejercicio 10:}

A continuación agregamos la regla de producción para el comando \textbf{for} a la
gramática y extendemos la semántica operacional.

\subsection*{Regla de producción en la gramática abstracta de LIS}

\begin{align*}
comm ::=& \; \textbf{skip} \\
        &| \;\; var = intexp \\
        &| \;\; comm; \; comm \\
        &| \;\; \textbf{if} \; boolexp \; \textbf{then} \; comm \; \textbf{else} \; comm \\
        &| \;\; \textbf{while} \; boolexp \; \textbf{do} \; comm \\
        &| \;\; \textbf{for} \; (intexp \; ; \; boolexp \; ; \; intexp) \; comm  
\end{align*}

\subsection*{Semántica operacional para el comando for}
\[
\infr[\rn{For}]
  {\sexp
    {\pair{e_1}{\sigma}}
    {\pair{n}{\sigma'}}
  }
  {\step
    {\pair{\cfor{e_1}{e_2}{e_3}{c}}{\sigma}}
    {\pair{\cwhile{e_2}{(\cseq{c}{{\cif{e_3 == 0}{\cskip}{\cskip}}})}}
          {\sigma'}}
  }
\
\]

\end{document}
\documentclass[11pt]{article}
\usepackage[a4paper, margin=2.54cm]{geometry}
\usepackage[utf8]{inputenc}
\usepackage[spanish, mexico]{babel}
\usepackage[spanish]{layout}
\usepackage[article]{ragged2e}
\usepackage{textcomp}
\usepackage{amsmath}
\usepackage{amssymb}
\usepackage{amsfonts}
\usepackage{proof}

\setlength{\parindent}{5pt}

\title{Trabajo Práctico: Unidad 6}
\author{Mellino, Natalia \and Farizano, Juan Ignacio}
\date{}

\begin{document}
\maketitle

%==============================================================================
%==============================================================================
%==============================================================================

\section{Ejercicio 1}
\subsection{Sintaxis Abstracta}

\begin{align*}
intexp ::&= nat \; | \; var \; | \; -_u \; intexp \\
         &| \;\; intexp \; +  \; intexp \\
         &| \;\; intexp \; -_b  \; intexp \\
         &| \;\; intexp \; \times  \; intexp \\
         &| \;\; intexp \; \div  \; intexp \\
         &| \;\; var = intexp \\
         &| \;\; intexp, \; intexp \\
boolesxp ::&= \textbf{true} \; | \; \textbf{false} \\
           &| \;\; intexp \; ==  \; intexp \\
           &| \;\; intexp \; \neq  \; intexp \\
           &| \;\; intexp \; <  \; intexp \\
           &| \;\; intexp \; >  \; intexp \\
           &| \;\; boolexp \; \lor  \; boolexp \\
           &| \;\; boolexp \; \land  \; boolexp \\
           &| \;\; \neg \; boolexp \\
comm ::&= \textbf{skip} \\
       &| \;\; var = intexp \\
       &| \;\; comm; \; comm \\
       &| \;\; \textbf{if} \; boolexp \; \textbf{then} \; comm \; \textbf{else} \; comm \\
       &| \; \; \textbf{while} \; boolexp \; \textbf{do} \; comm
\end{align*}

%==============================================================================

\subsection{Sintaxis Concreta}

\begin{align*}
  digit ::&= \text{'0'} \; | \; \text{'1'} | \; \dotsc \; | \; \text{'9'} \\
  letter ::&= \text{'a'} \; | \; \dotsc \; | \; \text{'Z'} \\
  nat ::&= digit \; | \; digit \; nat \\
  var ::&= letter \; | \; letter \; var \\
  intexp ::&= nat \\
           &| \; var \\
           &| \; \text{'-'} \; intexp \\
           &| \;\; intexp \; \text{'+'}  \; intexp \\
           &| \;\; intexp \; \text{'-'}  \; intexp \\
           &| \;\; intexp \; \text{'*'}  \; intexp \\
           &| \;\; intexp \; \text{'/'}  \; intexp \\
           &| \;\; \text{'('} \; intexp \; \text{')'} \\
           &| \;\; var \; \text{'='} \; intexp \\
           &| \;\; intexp \; \text{','} \; intexp \\
boolesxp ::&= \text{'true'} \; | \; \text{'false'} \\
           &| \;\; intexp \; \text{'=='} \; intexp \\
           &| \;\; intexp \; \text{'!='}  \; intexp \\
           &| \;\; intexp \; \text{'$<$'}  \; intexp \\
           &| \;\; intexp \; \text{'$>$'}  \; intexp \\
           &| \;\; boolexp \; \text{'\&\&'}  \; boolexp \\
           &| \;\; boolexp \; \text{'$\vert\vert$'}  \; boolexp \\
           &| \;\; \text{'!'} \; boolexp \\
           &| \;\; \text{'('} \; boolexp \; \text{')'} \\
com ::&= \textbf{skip} \\
      &| \;\; var \; \text{'='} \; intexp \\
      &| \;\; comm \; \text{';'} \; comm \\
      &| \;\; \text{'if'} \;\;boolexp \;\; \text{'\{'} \;\; comm \;\; \text{'\}'} \\
      &| \;\; \text{'if'} \;\;boolexp \;\; \text{'\{'} \;\; comm \;\; \text{'\}'}  \;\; \text{'else'} \;\; \text{'\{'} \;\; comm \;\; \text{'\}'} \\
      &| \;\; \text{'while'} \;\;boolexp \;\; \text{'\{'} \;\; comm \;\; \text{'\}'} \\
\end{align*}

%==============================================================================
%==============================================================================
%==============================================================================

\section{Ejercicio 4:}

%==============================================================================
%==============================================================================
%==============================================================================

\section{Ejercicio 5:}

Asumimos que la relación $\Downarrow_{exp}$ es determinista y procedemos por inducción
en la última regla de derivación.
Queremos probar : $ c \rightsquigarrow c', \: c \rightsquigarrow c'' \Rightarrow c' = c'' $ \\

$\bullet$ Si $ c \rightsquigarrow c' $ usando como última regla \emph{ASS}: \\
$c$ tiene la forma $\langle v=e, \sigma \rangle$ y tenemos la premisa
$\langle e, \sigma \rangle \: \Downarrow_{exp} \: \langle n, \sigma' \rangle$,
inmediatamente debido a la regla \emph{ASS} obtenemos que
$c' = \: \langle \textbf{skip}, [\sigma' \: \vert v : n] \rangle$.

Supongamos entonces, que esta relación no es determinista, es decir que $ c' \neq c'' $.
Por la forma que tiene $c$ observemos que la única regla que podemos usar en la
derivación $ c \rightsquigarrow c'' $ es la regla \emph{ASS}, entonces tenemos que:
$ \langle v=e, \sigma \rangle \rightsquigarrow \langle \textbf{skip}, [\sigma'' \: \vert v : n'] \rangle$
con la premisa $\langle e, \sigma \rangle \: \Downarrow_{exp} \: \langle n', \sigma'' \rangle$
Como $ c' \neq c'' $ vemos que $ \sigma' \neq \sigma'' \lor n \neq n''$.
Esto es una contradicción ya que por determinismo de la relación $\Downarrow_{exp}$ 
necesariamente debe ocurrir que $ \sigma' = \sigma'' \land n = n''$. \\

$\therefore \; c' = c''$



%==============================================================================

\end{document}
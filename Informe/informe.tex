\documentclass[11pt, fleqn]{article}
\usepackage[a4paper, margin=2.54cm]{geometry}
\usepackage[utf8]{inputenc}
\usepackage[spanish, mexico]{babel}
\usepackage[spanish]{layout}
\usepackage[article]{ragged2e}
\usepackage{textcomp}
\usepackage{amsmath}
\usepackage{amssymb}
\usepackage{amsfonts}
\usepackage{proof}

\setlength{\parindent}{5pt}

\title{Trabajo Práctico 1}
\author{Mellino, Natalia \and Farizano, Juan Ignacio}
\date{}

\begin{document}
\maketitle

%==============================================================================
%==============================================================================
%==============================================================================

\section*{Ejercicio 1}
A continuación extendemos las sintaxis abstracta y concreta para incluir asignaciones
de variables como expresiones enteras y para escribir una secuencia de expresiones enteras.

\subsection*{Sintaxis Abstracta}
\begin{align*}
intexp ::=& \; nat \; | \; var \; | \; -_u \; intexp \\
          &| \;\; intexp \; +  \; intexp \\
          &| \;\; intexp \; -_b  \; intexp \\
          &| \;\; intexp \; \times  \; intexp \\
          &| \;\; intexp \; \div  \; intexp \\
          &| \;\; var = intexp \\
          &| \;\; intexp, \; intexp \\
boolesxp ::=& \; \textbf{true} \; | \; \textbf{false} \\
            &| \;\; intexp \; ==  \; intexp \\
            &| \;\; intexp \; \neq  \; intexp \\
            &| \;\; intexp \; <  \; intexp \\
            &| \;\; intexp \; >  \; intexp \\
            &| \;\; boolexp \; \lor  \; boolexp \\
            &| \;\; boolexp \; \land  \; boolexp \\
            &| \;\; \neg \; boolexp \\
comm ::=& \; \textbf{skip} \\
        &| \;\; var = intexp \\
        &| \;\; comm; \; comm \\
        &| \;\; \textbf{if} \; boolexp \; \textbf{then} \; comm \; \textbf{else} \; comm \\
        &| \; \; \textbf{while} \; boolexp \; \textbf{do} \; comm
\end{align*}

%==============================================================================

\subsection*{Sintaxis Concreta}

\begin{align*}
digit ::=& \; \text{'0'} \; | \; \text{'1'} | \; \dotsc \; | \; \text{'9'} \\
letter ::=& \; \text{'a'} \; | \; \dotsc \; | \; \text{'Z'} \\
nat ::=& \; digit \; | \; digit \; nat \\
var ::=& \; letter \; | \; letter \; var \\
intexp ::=& \; nat \\
          &| \; var \\
          &| \; \text{'-'} \; intexp \\
          &| \;\; intexp \; \text{'+'}  \; intexp \\
          &| \;\; intexp \; \text{'-'}  \; intexp \\
          &| \;\; intexp \; \text{'*'}  \; intexp \\
          &| \;\; intexp \; \text{'/'}  \; intexp \\
          &| \;\; \text{'('} \; intexp \; \text{')'} \\
          &| \;\; var \; \text{'='} \; intexp \\
          &| \;\; intexp \; \text{','} \; intexp \\
boolesxp ::=& \; \text{'true'} \; | \; \text{'false'} \\
            &| \;\; intexp \; \text{'=='} \; intexp \\
            &| \;\; intexp \; \text{'!='}  \; intexp \\
            &| \;\; intexp \; \text{'$<$'}  \; intexp \\
            &| \;\; intexp \; \text{'$>$'}  \; intexp \\
            &| \;\; boolexp \; \text{'\&\&'}  \; boolexp \\
            &| \;\; boolexp \; \text{'$\vert\vert$'}  \; boolexp \\
            &| \;\; \text{'!'} \; boolexp \\
            &| \;\; \text{'('} \; boolexp \; \text{')'} \\
com ::=& \; \textbf{skip} \\
       &| \;\; var \; \text{'='} \; intexp \\
       &| \;\; comm \; \text{';'} \; comm \\
       &| \;\; \text{'if'} \;\;boolexp \;\; \text{'\{'} \;\; comm \;\; \text{'\}'} \\
       &| \;\; \text{'if'} \;\;boolexp \;\; \text{'\{'} \;\; comm \;\; \text{'\}'}  \;\; \text{'else'} \;\; \text{'\{'} \;\; comm \;\; \text{'\}'} \\
       &| \;\; \text{'while'} \;\;boolexp \;\; \text{'\{'} \;\; comm \;\; \text{'\}'} \\
\end{align*}

%==============================================================================
%==============================================================================
%==============================================================================

\section*{Ejercicio 2:}
Para extender la realización de la sistaxis abstracta en Haskell, incluimos
estos contructores en el tipo de datos parametrizado Exp $a$. 
\begin{align*}
      \text{EAssgn} &:: \text{Variable} \rightarrow \text{Exp Int} \rightarrow \text{Exp Int} \\
      \text{ESeq} &:: \text{Exp Int} \rightarrow \text{Exp Int} \rightarrow \text{Exp Int} \\
\end{align*}
En el archivo \emph{src/AST.hs} se encuentra reflejado este cambio.

%==============================================================================
%==============================================================================
%==============================================================================

\section*{Ejercicio 3:}
Para implementar el parser, modificamos la gramática extendida en el ejercicio 1
en una que no presente ambigüedad.

\subsection*{Sintaxis Abstracta}
\begin{align*}
intexp ::=& \; intexp, \; intexp1 \; \vert \; intexp1 \\
intexp1 ::=& \; var = intexp1 \; \vert \; intexp2 \\
intexp2 ::=& \; intexp2 \; +  \; intexp3 \; \vert \; intexp2 \; -_b  \; intexp3 \; \vert \; intexp2\\
intexp3 ::=& \; intexp3 \; \times  \; intexp4 \; \vert \; intexp3 \; \div  \; intexp4 \; \vert \; intexp4\\
intexp4 ::=& \; -_u \; intexp4 \; \vert \; nat \; \vert \; var \; \vert \; (intexp) \\
& \\
boolexp ::=& \; boolexp \; \lor \; boolexp1 \; \vert \; boolexp1 \\
boolexp1 ::=& \; boolexp1 \; \land \; boolexp2 \; \vert \; boolexp2 \\
boolexp2 ::=& \; \neg \; boolexp2 \; \vert \; boolexp3 \\
boolexp3 ::=& \; \textbf{true} \; \vert \; \textbf{false} \\
            &| \;\; intexp == intexp \\
            &| \;\; intexp \neq intexp \\
            &| \;\; intexp > intexp \\
            &| \;\; intexp < intexp \\
            &| \;\; (boolexp) \\
&\\
comm ::=& \; comm; \; comm1 \; \vert \; comm1 \\
comm1 ::=& \; \textbf{skip} \\
         &| \;\; var = intexp \\
         &| \;\; \textbf{if} \; boolexp \; \textbf{then} \; comm \; \textbf{else} \; comm \\
         &| \; \; \textbf{while} \; boolexp \; \textbf{do} \; comm
\end{align*}

%==============================================================================

\subsection*{Sintaxis Concreta}

\begin{align*}
digit ::=& \; \text{'0'} \; | \; \text{'1'} | \; \dotsc \; | \; \text{'9'} \\
letter ::=& \; \text{'a'} \; | \; \dotsc \; | \; \text{'Z'} \\
nat ::=& \; digit \; | \; digit \; nat \\
var ::=& \; letter \; | \; letter \; var \\
&\\
intexp ::=& \; intexp \; \text{','} \; intexp1 \; \vert \; intexp1 \\
intexp1 ::=& \; var \; \text{'='} \; intexp1 \; \vert \; intexp2 \\
intexp2 ::=& \; intexp2 \; \text{'+'}  \; intexp3 \; \vert \; intexp2 \; \text{'-'} \; intexp3 \; \vert \; intexp2\\
intexp3 ::=& \; intexp3 \; \text{'*'}  \; intexp4 \; \vert \; intexp3 \; \text{'/'} \; intexp4 \; \vert \; intexp4\\
intexp4 ::=& \; \text{'-'} \; intexp4 \; \vert \; nat \; \vert \; var \; \vert \; \text{'('} intexp \text{')'} \\
& \\
boolexp ::=& \; boolexp \; \text{'$\vert\vert$'} \; boolexp1 \; \vert \; boolexp1 \\
boolexp1 ::=& \; boolexp1 \; \text{'\&\&'} \; boolexp2 \; \vert \; boolexp2 \\
boolexp2 ::=& \; \text{'!='} \; boolexp2 \; \vert \; boolexp3 \\
boolexp3 ::=& \; \textbf{true} \; \vert \; \textbf{false} \\
            &| \;\; intexp \; \text{'=='} \; intexp \\
            &| \;\; intexp \; \text{'!='} \; intexp \\
            &| \;\; intexp \; \text{'$<$'} \; intexp \\
            &| \;\; intexp \; \text{'$>$'} \; intexp \\
            &| \;\; (boolexp) \\
&\\
comm ::=& \; comm \; \text{';'} \; comm1 \; \vert \; comm1 \\
comm1 ::=& \; \textbf{skip} \\
         &| \;\; var \; \text{'='} \; intexp \\
         &| \;\; \text{'if'} \;\;boolexp \;\; \text{'\{'} \;\; comm \;\; \text{'\}'} \\
         &| \;\; \text{'if'} \;\;boolexp \;\; \text{'\{'} \;\; comm \;\; \text{'\}'}  \;\; \text{'else'} \;\; \text{'\{'} \;\; comm \;\; \text{'\}'} \\
         &| \;\; \text{'while'} \;\;boolexp \;\; \text{'\{'} \;\; comm \;\; \text{'\}'} \\
\end{align*}

%==============================================================================
%==============================================================================
%==============================================================================

\section*{Ejercicio 4:}

%==============================================================================
%==============================================================================
%==============================================================================

\section*{Ejercicio 5:}

Asumimos que la relación $\Downarrow_{exp}$ es determinista y procedemos por inducción
en la última regla de derivación.
Queremos probar : $ c \rightsquigarrow c', \: c \rightsquigarrow c'' \Rightarrow c' = c'' $ \\

$\bullet$ Si $ c \rightsquigarrow c' $ usando como última regla \emph{ASS}: \\
$c$ tiene la forma $\langle v=e, \sigma \rangle$ y tenemos la premisa
$\langle e, \sigma \rangle \: \Downarrow_{exp} \: \langle n, \sigma' \rangle$,
inmediatamente debido a la regla \emph{ASS} obtenemos que
$c' = \: \langle \textbf{skip}, [\sigma' \: \vert v : n] \rangle$.

Supongamos entonces, que esta relación no es determinista, es decir que $ c' \neq c'' $.
Por la forma que tiene $c$ observemos que la única regla que podemos usar en la
derivación $ c \rightsquigarrow c'' $ es la regla \emph{ASS}, entonces tenemos que:
$ \langle v=e, \sigma \rangle \rightsquigarrow \langle \textbf{skip}, [\sigma'' \: \vert v : n'] \rangle$
con la premisa $\langle e, \sigma \rangle \: \Downarrow_{exp} \: \langle n', \sigma'' \rangle$
Como $ c' \neq c'' $ vemos que $ \sigma' \neq \sigma'' \lor n \neq n''$.
Esto es una contradicción ya que por determinismo de la relación $\Downarrow_{exp}$ 
necesariamente debe ocurrir que $ \sigma' = \sigma'' \land n = n''$. \\

$\therefore \; c' = c''$ \\


%==============================================================================

$\bullet$ Si $ c \rightsquigarrow c' $ usando como última regla \emph{SEQ1}: \\
$c$ tiene la forma $\langle \textbf{skip};c_1, \sigma \rangle$ y tenemos que $c' = \langle c_1, sigma \rangle$.
Por la forma que tiene $c$, observemos que no es posible aplicar ninguna otra regla
de inferencia, por lo tanto en la derivación $ c \rightsquigarrow c'' $ se tiene que
$c'' = \langle c_1, \sigma \rangle$. \\

$\therefore \; c' = c''$ \\

%==============================================================================

$\bullet$ Si $ c \rightsquigarrow c' $ usando como última regla \emph{SEQ2}: \\
Tenemos entonces que $c$ tiene la forma: $\langle c_0;c_1, \sigma \rangle$ y
además:  $ \langle c_0, \; \sigma \rangle \rightsquigarrow \langle c_0', \sigma' \rangle$.
Ahora, analicemos que pasa en $ c \rightsquigarrow c'' $: ¿qué reglas podemos aplicar?

\begin{itemize}
      \item No podemos aplicar \emph{SEQ1} ya que esto nos diría que $c_0 = \textbf{skip}$ y esto
            nos contradice $ \langle c_0, \sigma \rangle \rightsquigarrow \langle c_0', \; \sigma \rangle$ ya que
            si $c_0 = skip$ el estado $\sigma$ no debería pasar a ser $\sigma'$.
      \item En cuanto a las demás reglas, es claro que por la forma que tiene $c$ no es
            posible aplicarlas.
\end{itemize}

Por lo tanto, la única regla que podemos aplicar es \emph{SEQ2}. Si suponemos que 
$c' \neq c''$ tenemos entonces que $c$ tiene la forma: $\langle c_0;c_1,\sigma \rangle$ y
además:  $ \langle c_0, \sigma \rangle \rightsquigarrow \langle c_0'', \sigma'' \rangle$.
Luego sigue que $c''$ tiene la forma: $ \langle c_0'';c1, \sigma'' \rangle$.
Sin embargo, observemos que $ \langle c_0, \sigma \rangle \rightsquigarrow \langle c_0', \; \sigma' \rangle$
es una subderivación y por lo tanto vale nuestra Hipótesis Inductiva, es decir necesariamente:

\begin{equation*}
 \langle c_0, \; \sigma \rangle \rightsquigarrow \langle c_0', \sigma' \rangle = \langle c_0, \; \sigma \rangle \rightsquigarrow \langle c_0'', \; \sigma'' \rangle
\end{equation*}

Por lo tanto, $c_0' = c_0''$ y $\sigma' = \sigma''$. \\

$\therefore \; c' = c''$ \\

%==============================================================================

$\bullet$ Si $ c \rightsquigarrow c' $ usando como última regla \emph{IF1}: \\
$c$ tiene la forma $ \langle \textbf{if } \text{b \textbf{then} } c_0 \textbf{ else } c_1, \; \sigma \rangle$
y además $ \langle b, \sigma \rangle \Downarrow_{exp} \langle \textbf{true}, \sigma' \rangle$.
Entonces $c'$ tiene la forma $ \langle c_0, \sigma' \rangle$. ¿Qué pasa en el caso
de  $ c \rightsquigarrow c'' $. Sabemos que por la forma que tiene $c$ y por el hecho de
que $ \langle b, \sigma \rangle \Downarrow_{exp} \langle \textbf{true}, \sigma' \rangle$
sólo podemos aplicar la regla \emph{IF1}. Recordemos que la relación $\Downarrow_{exp}$ es
determinista y por eso podemos decir que sólo podamos aplicar \emph{IF1}. inmediatamente
de esta conclusión surge que $c' = c''$. \\

%==============================================================================

$\bullet$ Si $ c \rightsquigarrow c' $ usando como última regla \emph{IF2}:
\begin{center}(este caso es análogo a la regla \emph{IF1})\end{center}

%==============================================================================

$\bullet$ Si $ c \rightsquigarrow c' $ usando como última regla \emph{WHILE1}: tenemos entonces que
\begin{itemize}
      \item $c$ tiene la forma $ \langle \textbf{while } \text{b \textbf{do} } c_0, \sigma \rangle$.
      \item $ \langle b, \sigma \rangle \Downarrow_{exp} \langle \textbf{true}, \sigma' \rangle$
      \item $c' = \langle c0; \textbf{while } \text{b \textbf{do} } c_0, \sigma' \rangle$
\end{itemize}

Luego en $ c \rightsquigarrow c'' $ ocurre que por la forma de $c$ y por el determinismo
de $\Downarrow_{exp}$, la única regla que podemos usar es \emph{WHILE1}. Por estas dos cosas
surge inmediatamente que $c' = c''$. \\

%==============================================================================

$\bullet$ Si $ c \rightsquigarrow c' $ usando como última regla \emph{WHILE2}: tenemos entonces que
\begin{itemize}
      \item $c$ tiene la forma $ \langle \textbf{while } \text{b \textbf{do} } c_0, \; \sigma \rangle$.
      \item $ \langle b, \sigma \rangle \Downarrow_{exp} \langle \textbf{false}, \sigma' \rangle$
      \item $c' = \langle \textbf{skip}, \sigma' \rangle$
\end{itemize}

Luego en $ c \rightsquigarrow c'' $ ocurre que por la forma de $c$ y por el determinismo
de $\Downarrow_{exp}$, la única regla que podemos usar es \emph{WHILE1}. Por estas dos cosas
surge inmediatamente que $c' = c''$. \\

Por lo tanto, concluimos que la relación $ \rightsquigarrow $ es determinista.

%==============================================================================
%==============================================================================
%==============================================================================

\section*{Ejercicio 6:}

\end{document}
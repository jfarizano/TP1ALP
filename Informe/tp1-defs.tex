% Macros para type-setear el Lenguaje Imperativo Simple del TP1 de ALP.

\usepackage{amsmath, amssymb, proof}

%%%%%%%%%%%%%%%%%%%%%%%%%%%%%%%%%%%%%%%%%%%%%%%%%%%%%%%%%%%%%%%%%%%%%%%%%%
% Sintaxis Abstracta LIS
%%%%%%%%%%%%%%%%%%%%%%%%%%%%%%%%%%%%%%%%%%%%%%%%%%%%%%%%%%%%%%%%%%%%%%%%%%

%% Expresiones enteras
\newcommand{\euminus}   [1]  {-_{u} #1}
\newcommand{\eplus}     [2]  {#1 + #2}
\newcommand{\eminus}    [2]  {#1 -_{b} #2}
\newcommand{\etimes}    [2]  {#1 \times #2}
\newcommand{\ediv}      [2]  {#1 \div #2}
\newcommand{\eseq}      [2]  {#1 ,~ #2}
\newcommand{\eassgn}    [2]  {#1 = #2}

%% Expresiones booleanas
\newcommand{\etrue}          {\textbf{true}}
\newcommand{\efalse}         {\textbf{false}}
\newcommand{\eeq}       [2]  {#1 == #2}
\newcommand{\eneq}      [2]  {#1 \neq #2}
\newcommand{\elt}       [2]  {#1 < #2}
\newcommand{\egt}       [2]  {#1 > #2}
\newcommand{\eand}      [2]  {#1 \wedge #2}
\newcommand{\eor}       [2]  {#1 \vee #2}
\newcommand{\enot}      [1]  {\neg #1}

%% Comandos
\newcommand{\cskip}          {\textbf{skip}}
\newcommand{\clet}      [2]  {#1 = #2}
\newcommand{\cseq}      [2]  {#1;~#2}
\newcommand{\cif}       [3]  {\textbf{if}~#1~\textbf{then}~#2~\textbf{else}~#3}
\newcommand{\cwhile}    [2]  {\textbf{while}~#1~\textbf{do}~#2}
\newcommand{\cfor}      [4]  {\textbf{for}(#1;~#2~;#3)~#4}

%%%%%%%%%%%%%%%%%%%%%%%%%%%%%%%%%%%%%%%%%%%%%%%%%%%%%%%%%%%%%%%%%%%%%%%%%%
% Operadores del metalenguaje (de enteros y booleanos)
%%%%%%%%%%%%%%%%%%%%%%%%%%%%%%%%%%%%%%%%%%%%%%%%%%%%%%%%%%%%%%%%%%%%%%%%%%

%% Símbolos en negrita
\newcommand{\bs}        [1]  {\boldsymbol{#1}}

%% Números en negrita
\newcommand{\bm}        [1]  {\mathbf{#1}}

%% Operaciones sobre enteros
\newcommand{\muminus}   [1]  {\bs{-} #1}
\newcommand{\mplus}     [2]  {#1 \bs{+} #2}
\newcommand{\mminus}    [2]  {#1 \bs{-} #2}
\newcommand{\mtimes}    [2]  {#1 \bs{\times} #2}
\newcommand{\mdiv}      [2]  {#1 \bs{\div} #2}

%% Operaciones sobre booleanos
\newcommand{\mtrue}          {\textbf{true}}
\newcommand{\mfalse}         {\textbf{false}}
\newcommand{\meq}       [2]  {#1 \bs{=} #2}
\newcommand{\mneq}      [2]  {#1 \bs{\neq} #2}
\newcommand{\mlt}       [2]  {#1 \bs{<} #2}
\newcommand{\mgt}       [2]  {#1 \bs{>} #2}
\newcommand{\mand}      [2]  {#1 \bs{\wedge} #2}
\newcommand{\mor}       [2]  {#1 \bs{\vee} #2}
\newcommand{\mnot}      [1]  {\bs{\neg} #1}

%%%%%%%%%%%%%%%%%%%%%%%%%%%%%%%%%%%%%%%%%%%%%%%%%%%%%%%%%%%%%%%%%%%%%%%%%%
% Resto del metalenguaje
%%%%%%%%%%%%%%%%%%%%%%%%%%%%%%%%%%%%%%%%%%%%%%%%%%%%%%%%%%%%%%%%%%%%%%%%%%

%% Pares de la forma <a, b>
\newcommand{\pair}      [2]  {\left\langle #1,~#2\right\rangle}

%% Relación semántica big-step de expresiones
\newcommand{\sexp}      [2]  {#1 \Downarrow_{\mathrm{exp}} #2}

%% Relación semántica small-step de expresiones
\newcommand{\step}      [2]  {#1 \rightsquigarrow #2}

%% Clausura reflexo-transitiva de \step
\newcommand{\steps}     [2]  {#1 \rightsquigarrow^{*} #2}

%% Extensión funcional para estado - [f| x: e]
\newcommand{\extf}      [3]  {\left[ #1|~#2:~#3 \right]}

%% Regla de inferencia, con argumentos invertidos
%% \infr{L}{A}{B} =     A
%%                  --------- L
%%                      B
\newcommand{\infr}    [3] [] {\infer[\textsc{#1}]{#3}{#2}}

%% Estilo para nombre de raglas
\newcommand{\rn}        [1]  {\textsc{#1}}

%%%%%%%%%%%%%%%%%%%%%%%%%%%%%%%%%%%%%%%%%%%%%%%%%%%%%%%%%%%%%%%%%%%%%%%%%%
% Nombre de reglas
%%%%%%%%%%%%%%%%%%%%%%%%%%%%%%%%%%%%%%%%%%%%%%%%%%%%%%%%%%%%%%%%%%%%%%%%%%
\newcommand{\rNVal}          {\rn{NVal}}
\newcommand{\rVar}           {\rn{Var}}
\newcommand{\rUMinus}        {\rn{UMinus}}
\newcommand{\rPlus}          {\rn{Plus}}
\newcommand{\rMinus}         {\rn{Minus}}
\newcommand{\rTimes}         {\rn{Times}}
\newcommand{\rDiv}           {\rn{Div}}
\newcommand{\rEq}            {\rn{Eq}}
\newcommand{\rNEq}           {\rn{NEq}}
\newcommand{\rGt}            {\rn{Gt}}
\newcommand{\rLt}            {\rn{Lt}}
\newcommand{\rBVal}          {\rn{BVal}}
\newcommand{\rNot}           {\rn{Not}}
\newcommand{\rOr}            {\rn{Or}}
\newcommand{\rAnd}           {\rn{And}}

\newcommand{\rAss}           {\rn{Ass}}
\newcommand{\rSeqA}          {\rn{Seq$_1$}}
\newcommand{\rSeqB}          {\rn{Seq$_2$}}
\newcommand{\rIfA}           {\rn{If$_1$}}
\newcommand{\rIfB}           {\rn{If$_2$}}
\newcommand{\rWhileA}        {\rn{While$_1$}}
\newcommand{\rWhileB}        {\rn{While$_2$}}